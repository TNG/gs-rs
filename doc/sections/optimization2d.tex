\section{Optimization in 2D}
\label{optimization_2d}
In the following sections, the individual 2D factors' calculations of $J$ and $e$ are presented. The functions $pos(x)$ and $rot(x)$ will be used to refer to the 2D position vector and the rotation angle of a 2D pose $x$, respectively. Similarly, the functions $pos_x(x)$ and $pos_y(x)$ will be used to refer to the single value within the respective dimension.

Furthermore, the rotation matrices are denoted such that
\begin{align}
	R_\alpha =
	\begin{pmatrix}
		cos(\alpha) & -sin(\alpha)\\
		sin(\alpha) &  cos(\alpha)
	\end{pmatrix}
\end{align}
equals the rotation matrix with the angle $\alpha$ in 2D, and
\begin{align}
	R_z(\alpha) =
	\begin{pmatrix}
		cos(\alpha) & -sin(\alpha) & 0\\
		sin(\alpha) &  cos(\alpha) & 0\\
		          0 &            0 & 1
	\end{pmatrix}
\end{align}
equals the rotation matrix with the angle $\alpha$ in 3D.



\subsection{Position2D}
The \textit{Position2D} factor involves one \textit{VehicleVariable} $x_v$. The Jacobian matrix $J$ in this case is
\begin{align}
	J = R_z(-rot(x_v)).
\end{align}
Given the measurement $x_m$, the error vector
\begin{align}
	e = R_{-rot(x_m)} * (pos(x_v) - pos(x_m))
\end{align}
can be computed as well.



\subsection{Odometry2D}
The \textit{Odometry2D} factor involves two \textit{VehicleVariables} $x_i$ and $x_j$. Given the measurement $x_{ij}$, the Jacobian matrix $J$ is calculated as follows:
\begin{align}
	\Delta x_{ij} = x_j - x_i
\end{align}
\begin{align}
	sin_i = sin(rot(x_i))
\end{align}
\begin{align}
	cos_i = cos(rot(x_i))
\end{align}
\begin{align}
	J_i = R_z(-rot(x_{ij})) *
	\begin{pmatrix}
		-cos_i & -sin_i & -sin_i*pos_x(\Delta x_{ij}) + cos_i*pos_y(\Delta x_{ij})\\
		 sin_i & -cos_i & -cos_i*pos_x(\Delta x_{ij}) - sin_i*pos_y(\Delta x_{ij})\\
		     0 &      0 &                                   -1
	\end{pmatrix}
\end{align}
\begin{align}
	J_j = R_z(-rot(x_{ij})) * R_z(-rot(x_i))
\end{align}
\begin{align}
	J =
	\begin{pmatrix}
		\boldsymbol{J_i} & \boldsymbol{J_j}
	\end{pmatrix}
\end{align}
The error vector $e$ is computed as follows:
\begin{align}
	e_{pos} = R_{-rot(x_{ij})} * (R_{-rot(x_i)} * pos(\Delta x_{ij}) - pos(x_{ij}))
\end{align}
\begin{align}
	e_{rot} = rot(\Delta x_{ij}) - rot(x_{ij})
\end{align}
After normalizing $e_{rot}$ to $[-\pi, \pi)$ with $norm(e_{rot})$ the full error vector can be constructed with
\begin{align}
	e =
	\begin{pmatrix}
		\boldsymbol{e_{pos}}\\
		norm(e_{rot})
	\end{pmatrix}
	.
\end{align}



\subsection{Observation2D}
The \textit{Observation2D} factor involves one \textit{VehicleVariable} $x_i$ and one \textit{LandmarkVariable} $x_j$. The measurement is denoted as $x_{ij}$, analogously to the previous section. Although $x_j$ and $x_{ij}$ are only positions rather than poses and therefore do not contain a rotation angle, the functions $pos(x)$, $pos_x(x)$ and $pos_y(x)$ will be used nevertheless to make the calculation path more understandable. Given the measurement $x_{ij}$, the Jacobian matrix $J$ is calculated as follows:
\begin{align}
	pos(\Delta x_{ij}) = pos(x_j) - pos(x_i)
\end{align}
\begin{align}
	sin_i = sin(rot(x_i))
\end{align}
\begin{align}
	cos_i = cos(rot(x_i))
\end{align}
\begin{align}
	J_i =
	\begin{pmatrix}
		-cos_i & -sin_i & -sin_i*pos_x(\Delta x_{ij}) + cos_i*pos_y(\Delta x_{ij})\\
		 sin_i & -cos_i & -cos_i*pos_x(\Delta x_{ij}) - sin_i*pos_y(\Delta x_{ij})
	\end{pmatrix}
\end{align}
\begin{align}
	J_j = R_{-rot(x_i)}
\end{align}
\begin{align}
	J =
	\begin{pmatrix}
		\boldsymbol{J_i} & \boldsymbol{J_j}
	\end{pmatrix}
\end{align}
The error vector $e$ is computed as follows:
\begin{align}
	e = R_{-rot(x_i)} * pos(\Delta x_{ij}) - pos(x_{ij})
\end{align}